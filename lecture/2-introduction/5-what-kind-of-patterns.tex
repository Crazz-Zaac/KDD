\section{What kind of patterns can be mined?}

\begin{frame}{What kind of patterns can be mined?}
	\begin{itemize}
		\item \textbf{Searching for the right patterns is important.}
		\item Which patterns can be mined depends on:
		\begin{itemize}
			\item \textbf{The data mining function} \\
				  \small{Different functions can reveal different patterns.}
			\item \textbf{The data set} \\
				  \small{Some types of records contain special patterns that 
				  can be found only in them.}
		\end{itemize}
		\item Patterns do not always lead to useful information. \\
				$\rightarrow$ Always validate whether the gained knowledge is 
				interesting
	\end{itemize}
\end{frame}

\begin{frame}{Data Mining Function: I. Generalization}
	\textbf{Information integration and data warehouse construction:}
	\begin{itemize}
		\item Data cleaning.
		\item Transformation.
		\item Integration.
		\item Multidimensional modeling.
	\end{itemize}
	\textbf{Data cube technology:}
	\begin{itemize}
		\item Characterization (contrast data characteristics).\\
		E.g. dry vs. wet regions from numerical humidity values.
		\item Discrimination.
		\item Generalization.
		\item Summary.
	\end{itemize}
\end{frame}

\begin{frame}{Data Mining Function: II. Association and Correlation Analysis}
	\textbf{Frequent patterns or item sets:}\\
	What items are frequently purchased together in your supermarket.\\[0.5cm]
	
	\textbf{Association, correlation vs. causality:}\\
	A typical association rule: Diapers $\rightarrow$ Beer $[0.5\%,75\%]$ 
	(support, confidence).\\
	Are strongly associated items also strongly correlated?\\[0.5cm]
	
	\textbf{How to mine such patterns and rules efficiently in large 
	datasets?}\\
	\textbf{How to use such patterns for classification, clustering and other 
	applications?}
\end{frame}

\begin{frame}{Data Mining Function: III. Classification}
	\textbf{Classification and (class-)label prediction:}\\
	Construct models (functions) based on training examples. \\
	Hence: "supervised".\\
	Describe and distinguish classes or concepts for future prediction.\\
	E.g. classify countries based on climate or classify cars based on gas 
	mileage.\\
	Classifying something means to predict unknown class labels. \\[0.5cm]
	
	\textbf{Typical methods:}\\
	Decision trees, naive Bayesian classification, support-vector machines, 
	neural networks, rule-based classification, pattern-based classification, 
	logistic regression \ldots\\[0.5cm]
	
	\textbf{Typical applications:}\\
	Credit-card-fraud detection, direct marketing, classifying stars, diseases, 
	web pages \ldots
\end{frame}

\begin{frame}{Data Mining Function: IV. Cluster Analysis}
	\textbf{Unsupervised learning:} I.e. class labels are unknown.\\
	\textbf{Group data:} I.e. cluster houses to find distribution 
	patterns.\\[0.5cm]
	
	Principle:\\
	Maximize intra class similarity and minimize inter class 
	similarity.\\[0.5cm]
	
	What is \textbf{similarity?}
\end{frame}

\begin{frame}{Data Mining Function: V. Outlier Analysis}
	\textbf{Outlier}: A data object that does not comply with the general 
	behavior of the data.\\[0.5cm]
	
	Noise or exception?\\
	One person's garbage could be another person's treasure.\\[0.5cm]
	
	\textbf{Methods:}\\
	By-product of clustering or regression analysis \ldots \\
	Useful in fraud detection or rare-events analysis.
\end{frame}

\begin{frame}{Time and Ordering: Sequential Pattern, Trend and Evolution 
Analysis}
	\textbf{Sequence, trend, and evolution analysis}.\\
	\begin{itemize}
		\item Trend, time-series and deviation analysis. \\
		E.g., regression and value prediction (forecasting).
		\item Sequential-pattern mining.\\
		E.g. customers first buy a digital camera, then buy large SD memory 
		cards.
		\item Periodicity analysis.
		\item Motifs and biological-sequence analysis.\\
		Approximate and consecutive motifs.
		\item Similarity-based analysis.\\
		\item Mining data streams.\\
		Ordered, time-varying, potentially infinite (unbounded).
	\end{itemize}
\end{frame}

\begin{frame}{Structure and Network Analysis}
	\textbf{Graph mining}:\\
	Finding frequent subgraphs (e.g. chemical compounds), trees (XML), 
	substructures (web fragments), information-network analysis.\\[0.2cm]
	
	\textbf{Social networks}:
	\begin{itemize}
		\item Social networks: Actors (objects, nodes) and relationships 
		(edges).\\
		E.g., author networks in CS, terrorist networks.
		\item Multiple heterogeneous networks.\\
		A person could be in multiple information networks: friends, family, 
		classmates \ldots
		\item Links carry a lot of semantical information: link mining.
	\end{itemize}
	
	\textbf{Web mining}:
	\begin{itemize}
		\item Web is a big information network: from PageRank to Google.
		\item Analysis of web information networks.
		\item Web community discovery, opinion mining, usage mining \ldots
	\end{itemize}
\end{frame}

\begin{frame}{Evaluation of Knowledge}
	\textbf{Is all mined knowledge interesting?}
	\begin{itemize}
		\item One can mine tremendous amounts of "patterns" and knowledge.
		\item Some may fit only certain dimension space (time, location \ldots).
		\item Some may not be representative, may be transient \ldots
	\end{itemize}
	
	\textbf{Evaluation of mined knowledge $\rightarrow$ directly mine only 
	interesting knowledge?}
	\begin{itemize}
		\item Descriptive vs. predictive.
		\item Coverage.
		\item Typically vs. predictive.
		\item Accuracy.
		\item Timeliness.
		\item \ldots
	\end{itemize}
\end{frame}