% English version FAU Logo
\usepackage[english]{babel}
% German version FAU Logo
%\usepackage[ngerman]{babel}

\usepackage[utf8]{inputenc}
\usepackage[T1]{fontenc}
\usepackage{amsmath,amssymb}
\usepackage{graphicx}
\usepackage{listings}
\usepackage[backend=biber,sorting=none,doi=true,style=ieee]{biblatex}

% Options:
%  - inst:      Institute
%                 med:      MedFak FAU theme
%                 nat:      NatFak FAU theme
%                 phil:     PhilFak FAU theme
%                 rw:       RWFak FAU theme
%                 rw-jura:  RWFak FB Jura FAU theme
%                 rw-wiso:  RWFak FB WISO FAU theme
%                 tf:       TechFak FAU theme
%  - image:     Cover image on title page
%  - plain:     Plain title page
%  - longtitle: Title page layout for long title
\usetheme[%
  image,%
  longtitle,%
  inst=tf%
]{fau}

% Enable semi-transparent animation preview
\setbeamercovered{transparent}

% Enable frame numbering
%\setbeamertemplate{footline}[frame number]


\lstset{%
  language=C++,
  tabsize=2,
  basicstyle=\tt,
  keywordstyle=\color{blue},
  commentstyle=\color{green!50!black},
  stringstyle=\color{red},
  numbers=left,
  numbersep=0.5em,
  xleftmargin=1em,
  numberstyle=\tt
}

\defbibheading{bibliography}{}
\addbibresource[label=primary]{references.bib}

\date[SS2023]{Summer semester 2023}


\usepackage{url}
\usepackage{enumitem}
\usepackage{hyperref}
\usepackage{fontawesome5}
\usepackage{graphicx}
\usepackage{booktabs}
\usepackage{calc}
\usepackage{ifthen}
\usepackage{tikz}
\usepackage{tikz-cd}
\usepackage{pgfplots,pgfplotstable,pgf-pie}

\newcommand{\plots}{0.611201}
\newcommand{\plotm}{2.19882}
\pgfmathdeclarefunction{gauss}{2}{%
  \pgfmathparse{1/(#2*sqrt(2*pi))*exp(-((x-#1)^2)/(2*#2^2))}%
}

\tikzset{
  thick,
  >=latex,
  every edge/.style={draw=gray, thick, >=latex},
  vertex/.style = {
    circle,
    fill            = black,
    outer sep = 2pt,
    inner sep = 1pt,
  }
}
\usetikzlibrary{matrix,mindmap}
\usetikzlibrary{arrows,decorations.pathmorphing,backgrounds,fit,positioning,shapes.symbols,chains,intersections,snakes}
\tikzset{level 1/.append style={sibling angle=50,level distance = 165mm}}
\tikzset{level 2/.append style={sibling angle=20,level distance = 45mm}}
\tikzset{every node/.append style={scale=1}}
% read in data file
\pgfplotstableread{data/iris.dat}\iris
% get number of data points
\pgfplotstablegetrowsof{\iris}
\pgfmathsetmacro\NumRows{\pgfplotsretval-1}

\usepgfplotslibrary{groupplots}
\pgfplotsset{height=4cm,width=8cm,compat=1.14}
\newcommand{\tikzmark}[1]{\tikz[remember picture] \node[coordinate] (#1) {#1};}

% English version
\institute[CS6]{Chair of Computer Science 6 (Data Management), Friedrich-Alexander University Erlangen-N\"urnberg}
% German version
% \institute[Lehrstuhl]{Lehrstuhl, Friedrich-Alexander-Universit\"at Erlangen-N\"urnberg}

\setbeamertemplate{section in toc}[sections numbered]

\setbeamertemplate{section page}{%
    \begingroup
        \begin{beamercolorbox}[sep=10pt,center,rounded=true,shadow=true]{section title}
        \usebeamerfont{section title}\thesection~\insertsection\par
        \end{beamercolorbox}
    \endgroup
  }
