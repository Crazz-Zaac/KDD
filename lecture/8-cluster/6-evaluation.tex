\section{Evaluation of clustering}

\begin{frame}{Assessing Clustering Tendency}
	\begin{itemize}
		\item \textbf{Assess {\color{airforceblue}if non-random structure} 
		exists in the data by measuring the probability that the data is 
		generated by a uniform data distribution.}
		\item \textbf{Data with random structure:}
		\begin{itemize}
			\item Points uniformly distributed in data space.
		\end{itemize}
		\item \textbf{Clustering may return clusters, but:}
		\begin{itemize}
			\item Artificial partitioning.
			\item Meaningless.
		\end{itemize}
	\end{itemize}
\end{frame}

\begin{frame}{Hopkins Statistic (I)}
	\begin{itemize}
		\item \textbf{Data set:}
		\begin{itemize}
			\item Let $X = \{x_i \; \vert i=1,\ldots,n\}$ be a collection of n 
			patterns in a $d$-dimensional space such that $x_i = (x_{i1}, 
			x_{i2}, \ldots, x_{id})$.
		\end{itemize}
		\item \textbf{Random sample of data space:}
		\begin{itemize}
			\item Let $Y = \{y_j \; \vert \; j=1, \ldots, m\}$ be $m$ sampling 
			points placed at random in the $d$-dimensional space, with $m \ll 
			n$.
		\end{itemize}
		\item \textbf{Two types of distances defined:}
		\begin{itemize}
			\item $u_j$ as minimum distance from $y_j$ to its nearest pattern 
			in $X$ and
			\item $w_j$ as minimum distance from a randomly selected pattern in 
			$X$ to its nearest neighbor in $X$.
		\end{itemize}
		\item \textbf{The {\color{airforceblue}Hopkins} statistic in d 
		dimensions is defined as:}
		\begin{align}
			H = \frac{\sum_{j=1}^{m} u_j}{\sum_{j=1}^{m}u_j + \sum_{j=1}^{m} 
			w_j}. 
		\end{align}
	\end{itemize}
\end{frame}

\begin{frame}{Hopkins Statistic (II)}
	\begin{itemize}
		\item \textbf{Compares nearest-neighbor distribution of randomly 
		selected locations (points) to that for randomly selected patterns.}
		\item \textbf{Under the null hypothesis, $H_0$, of uniform 
		distribution:}
		\begin{itemize}
			\item Distances from sampling points to nearest patterns should, on 
			the average,\\ \textbf{\color{airforceblue}be the same} as the 
			interpattern nearest-neighbor distances, implying randomness.
			\item $H$ should be about $0.5$.
		\end{itemize}
		\item \textbf{When patterns are aggregated or clustered:}
		\begin{itemize}
			\item Distances from sampling points to nearest patterns should, on 
			the average,\\ \textbf{\color{airforceblue}larger} as the 
			interpattern nearest-neighbor distances.
			\item $H$ should be larger than $0.5$.
			\item Almost equal to 1.0 for very well clustered data.
		\end{itemize}
	\end{itemize}
\end{frame}

\begin{frame}{Determine the Number of Clusters}
	\begin{itemize}
		\item \textbf{Empirical method:}
		\begin{itemize}
			\item $\#$ of clusters $\approx \sqrt{\frac{n}{2}}$ for a dataset 
			of $n$ points.
		\end{itemize}
		\item \textbf{Elbow method:}
		\begin{itemize}
			\item Use the turning point in the curve of sum of within-cluster 
			variance w.r.t. the $\#$ of clusters.
		\end{itemize}
		\item \textbf{Cross-validation method:}
		\begin{itemize}
			\item Divide a given data set into $m$ parts.
			\item Use $m-1$ parts to obtain a clustering model.
			\item Use the remaining part to test the quality of the clustering.
			\begin{itemize}
				\item E.g., for each point in the test set, find the closest 
				centroid, and use the sum of squared distances between all 
				points in the test set and the closest centroids to measure how 
				well the model fits the test set.
			\end{itemize}
			\item For any $k > 0$, repeat it $m$ times, compare the overall 
			quality measure w.r.t. different $k$'s, and find $\#$ of clusters 
			that fits the data the best.
		\end{itemize}
	\end{itemize}
\end{frame}

\begin{frame}{Measuring Clustering Quality}
	\begin{itemize}
		\item \textbf{Two methods:}
		\begin{itemize}
			\item \textbf{{\color{airforceblue}Extrinsic}: supervised, i.e., 
			the ground truth is available.}
			\begin{itemize}
				\item Compare a clustering against the ground truth using 
				certain clustering quality measure.
				\begin{itemize}
					\item Ex. BCubed precision and recall metrics.
				\end{itemize}
			\end{itemize}
			\item \textbf{{\color{airforceblue}Intrinsic}: unsupervised, i.e., 
			the ground truth is unavailable.}
			\begin{itemize}
				\item Evaluate the goodness of a clustering by considering how 
				well the clusters are separated, and how compact the clusters 
				are.
				\begin{itemize}
					\item Ex. silhouette coefficient.
				\end{itemize}
			\end{itemize}
		\end{itemize}
	\end{itemize}
\end{frame}

\begin{frame}{Measuring Clustering Quality: Extrinsic Methods}
	\begin{itemize}
		\item \textbf{Clustering-quality measure: $Q(C,C_*)$.}
		\begin{itemize}
			\item For a clustering $C$ given the ground truth $C_*$.
		\end{itemize}
		\item \textbf{$Q$ is good, if it satisfies the following four essential 
		criteria:}
		\begin{itemize}
			\item Cluster homogeneity:
			\begin{itemize}
				\item The purer, the better.
			\end{itemize}
			\item Cluster completeness:
			\begin{itemize}
				\item Should assign objects that belong to the same category \\
				in the ground truth to the same cluster.
			\end{itemize}
			\item Rag bag:
			\begin{itemize}
				\item Putting a heterogeneous object into a pure cluster should 
				be penalized more than putting it into a rag bag (i.e., 
				"miscellaneous" or "other" category).
			\end{itemize}
			\item Small cluster preservation:
			\begin{itemize}
				\item Splitting a small category into pieces is more harmful 
				than \\
				splitting a large category into pieces.
			\end{itemize}
		\end{itemize}
	\end{itemize}
\end{frame}