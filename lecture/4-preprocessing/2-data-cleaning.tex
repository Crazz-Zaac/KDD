\section{Data cleaning}

\begin{frame}{Data Cleaning}
	\textbf{Data in the real world is {\color{airforceblue}dirty}. Lots of
		potentially incorrect data:}
	\begin{itemize}
		\item E.g. instrument faulty, human or computer error, transmission
		      error.
		\item \textbf{\color{airforceblue}Incomplete:} lacking attributes,
		      lacking certain attributes of interest or containing aggregate data.
		      \begin{itemize}
			      \item E.g. occupation = "" (missing data).
		      \end{itemize}
		\item \textbf{\color{airforceblue}Noisy:} containing noise.
		      \begin{itemize}
			      \item E.g. small measurement inaccuracies with a sensor
			            (noise)
		      \end{itemize}
		\item \textbf{\color{airforceblue}Errors/Outliers:} containing errors
		      or outliers.
		      \begin{itemize}
			      \item E.g. scores = "2,3,0,6,1,9,95" (outlier = "95")
			      \item E.g. salary = "-10" (error)
		      \end{itemize}
		\item \textbf{\color{airforceblue}Inconsistencies:} containing
		      discrepancies in codes or names.
		      \begin{itemize}
			      \item E.g. age = "42", birthday = "03/07/2010".
			      \item E.g. old rating = "1,2,3", new rating = "A,B,C".
			      \item E.g. discrepancy between duplicate records (e.g. address).
		      \end{itemize}
		\item \textbf{\color{airforceblue}Intentional} (only default value,
		      e.g. disguised missing data):
		      \begin{itemize}
			      \item E.g. "Doe" as everyone's surname
		      \end{itemize}
	\end{itemize}
\end{frame}

\begin{frame}{Incomplete (Missing) Data}
	\begin{itemize}
		\item \textbf{Data is not always available.}
		      \begin{itemize}
			      \item E.g. many tuples have no recorded value for several
			            attributes.
			      \item Examples are customer income in sales data.
		      \end{itemize}
		\item \textbf{Missing data may be due to:}
		      \begin{itemize}
			      \item Equipment malfunction.
			      \item Inconsistency with other recorded data and thus deleted.
			      \item Data not entered due to misunderstanding.
			      \item Certain data may not be considered important at the time of
			            entry.
			      \item Not registered history or changes of the data.
		      \end{itemize}
		\item \textbf{Missing data may need to be inferred.}
	\end{itemize}
\end{frame}

\begin{frame}{How to Handle Missing Data?}
	\begin{itemize}
		\item \textbf{Ignore the tuple:}
		      \begin{itemize}
			      \item Usually done when class label is missing (when doing
			            classification).
			      \item Not effective when the percentage of missing values per
			            attribute varies considerably.
		      \end{itemize}
		\item \textbf{Fill in the missing value manually.}
		      \begin{itemize}
			      \item Tedious or infeasible.
		      \end{itemize}
		\item \textbf{Fill in automatically with:}
		      \begin{itemize}
			      \item A global constant, e.g. "unkown", maybe a new class.
			      \item The attribute mean.
			      \item The attribute mean for all samples belonging to the same
			            class.
			      \item \textbf{\color{airforceblue} The most probable value:}
			            Inference-based such as Bayesian formula or decision tree.
		      \end{itemize}
	\end{itemize}
\end{frame}

\begin{frame}{Noisy Data}
	\begin{itemize}
		\item \textbf{\color{airforceblue}Noise:}
		      \begin{itemize}
			      \item Random error or variance in a measured variable.
			      \item Stored value a little bit off the real value, up or down.
			      \item Leads to (slightly) incorrect attribute values.
		      \end{itemize}
		\item \textbf{May be due to:}
		      \begin{itemize}
			      \item Faulty or imprecise data-collection instruments.
			      \item Data-entry problems.
			      \item Data-transmission problems.
			      \item Technology limitation.
			      \item Inconsistency in naming conventions.
		      \end{itemize}
	\end{itemize}
\end{frame}

\begin{frame}{How to Handle Noisy Data?}
	\begin{itemize}
		\item \textbf{Binning:}
		      \begin{itemize}
			      \item First sort data and partition into (equal-frequency) bins.
			      \item Then smooth by bin mean, by bin median or by bin boundaries.
		      \end{itemize}
		\item \textbf{Regression:}
		      \begin{itemize}
			      \item Smooth by fitting the data to regression functions.
		      \end{itemize}
		\item \textbf{Clustering:}
		      \begin{itemize}
			      \item Detect and remove outliers.
		      \end{itemize}
		\item \textbf{Combined computer and human inspection:}
		      \begin{itemize}
			      \item Detect suspicious values and check by human.
			      \item E.g. deal with possible outliers.
		      \end{itemize}
	\end{itemize}
\end{frame}

\begin{frame}{Data Cleaning as a Process (I)}
	\begin{itemize}
		\item \textbf{Data discrepancy detection:}
		      \begin{itemize}
			      \item Use \textbf{\color{airforceblue}metadata} (e.g. domain,
			            range, dependency, distribution).
			      \item Check field overloading.
			      \item Check uniqueness rule, consecutive rule and null rule.
			      \item Use commercial tools:
			            \begin{itemize}
				            \item \textbf{\color{airforceblue}Data scrubbing:} use simple
				                  domain knowledge (e.g. postal code, spell-check) to detect
				                  errors and make corrections.
				            \item \textbf{\color{airforceblue}Data auditing:} by analyzing
				                  data to discover rules and relationsships to detect violators
				                  (e.g. correlation and clustering to find outliers).
			            \end{itemize}
		      \end{itemize}
	\end{itemize}
\end{frame}

\begin{frame}{Data Cleaning as a Process (II)}
	\begin{itemize}
		\item \textbf{Data migration and integration:}
		      \begin{itemize}
			      \item Data-migration tools: allow transformations to be
			            specified.
			      \item ETL (Extraction/Transformation/Loading) tools: allow
			            users to specify transformations through a graphical user
			            interface.
		      \end{itemize}
		\item \textbf{Integration of the two processes.}
		      \begin{itemize}
			      \item Iterative and interactive (e.g. the Potter's Wheel tool).
		      \end{itemize}
	\end{itemize}
\end{frame}
