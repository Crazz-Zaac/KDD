\section{Data warehouse design and usage}

\begin{frame}{Design of Data Warehouse: A Business-analysis Framework}
  \begin{itemize}
  \item \textbf{Four views regarding the design of a data warehouse:}
    \begin{itemize}
    \item \textbf{\color{airforceblue}Top-down view:}
      \begin{itemize}
      \item Allows selection of the relevant information necessary for the data warehouse.
      \end{itemize}
    \item \textbf{\color{airforceblue}Data-source view:}
      \begin{itemize}
      \item Exposes the information being captured, stored, and managed by operational systems.
      \end{itemize}
    \item \textbf{\color{airforceblue}Data warehouse view:}
      \begin{itemize}
      \item Consists of fact tables and dimension tables.
      \end{itemize}
    \item \textbf{\color{airforceblue}Business-query view:}
      \begin{itemize}
      \item Sees the perspectives of data in the warehouse from the view of the end-user.
      \end{itemize}
    \end{itemize}
  \end{itemize}
\end{frame}

\begin{frame}{Data Warehouse Design Process}
  \begin{itemize}
  \item \textbf{Top-down, bottom-up approaches or a combination of both:}
    \begin{itemize}
    \item \textbf{\color{airforceblue}Top-down:} starts with overall design and planning (mature).
    \item \textbf{\color{airforceblue}Bottom-up:} starts with experiments and prototypes (rapid).
    \end{itemize}
  \item \textbf{From software-engineering point of view:}
    \begin{itemize}
    \item \textbf{\color{airforceblue}Waterfall:} structured and systematic analysis at each step before proceeding to the next.
    \item \textbf{\color{airforceblue}Spiral:} rapid generation of increasingly functional systems, short turn-around time.
    \end{itemize}
  \item \textbf{Typical Data warehouse design process:}
    \begin{itemize}
    \item Choose a \textbf{\color{airforceblue}business process} to model, e.g., orders, invoices, etc.
    \item Choose a \textbf{\color{airforceblue}grain} (atomic level of data) of the business process.
    \item Choose \textbf{\color{airforceblue}dimensions} that will apply to each fact-table record.
    \item Choose a \textbf{\color{airforceblue}measure} that will populate each fact-table record.
    \end{itemize}
  \end{itemize}
\end{frame}

\begin{frame}{Data Warehouse Development: A Recommended Approach}
  \centering
  \begin{tikzpicture}[>=latex']
    \tikzset{block/.style= {draw, rectangle, align=center,minimum width=2cm,minimum height=1cm},
      rblock/.style={draw, shape=rectangle,rounded corners=1.5em,align=center,minimum width=2cm,minimum height=1cm},
      input/.style={ % requires library shapes.geometric
        draw,
        trapezium,
        trapezium left angle=60,
        trapezium right angle=120,
        minimum width=2cm,
        align=center,
        minimum height=1cm
      },
    }
    \node[block] at (2,2) (ddm) {Distributed data marts};
    \node[block] at (8,3) (mtier) {Multi-tier data warehouse};

    \node[block] at (0.5,0) (datamart1) {Data mart};
    \node[block] at (3.5,0) (datamart2) {Data mart};
    \node[block] at (8,0) (edw) {Enterprise data warehouse};
    \node[block] at (5,-2) (hlw) {Define a high-level corporate data model};

    \path[draw,->] (hlw) edge (datamart1);
    \path[draw,->] (hlw) edge (datamart2);
    \path[draw,->] (datamart1) edge [bend right=30] node[fill=white] {Model refinement} (hlw);
    \path[draw,->] (datamart2) edge [bend left = 10] node[fill=white] {Model refinement} (hlw);
    \path[draw,->] (ddm) edge [bend left = 50] node[fill=white] {Model refinement} (hlw);
    \path[draw,->] (edw) edge [bend left = 40] node[fill=white] {Model refinement} (hlw);
    \path[draw,->] (datamart1) edge (ddm);
    \path[draw,->] (datamart2) edge (ddm);
    \path[draw,->] (ddm) edge (mtier);
    \path[draw,->] (edw) edge (mtier);
    \path[draw,->] (hlw) edge (edw);
  \end{tikzpicture}
\end{frame}

\begin{frame}{Data Warehouse Usage}
  \begin{itemize}
  \item \textbf{Three kinds of data warehouse applications.}
    \begin{itemize}
    \item \textbf{\color{airforceblue}Information processing.}
      \begin{itemize}
      \item Supports querying, basic statistical analysis, and \\ reporting using crosstabs, tables, charts and graphs.
      \end{itemize}
    \item \textbf{\color{airforceblue}Analytical processing.}
      \begin{itemize}
      \item Multidimensional analysis of data warehouse data.
      \item Supports basic OLAP operations, slice-dice, drilling, pivoting.
      \end{itemize}
    \item \textbf{\color{airforceblue}Data mining.}
      \begin{itemize}
      \item Knowledge discovery from hidden patterns.
      \item Supports associations, constructing analytical models, performing classification and prediction, and presenting the mining results using visualization tools.
      \end{itemize}
    \end{itemize}
  \end{itemize}
\end{frame}

\begin{frame}{From Online Analytical Processing (OLAP) To Online Analytical Mining (OLAM)}
  \begin{itemize}
  \item \textbf{Why online analytical mining?}
    \begin{itemize}
    \item DW contains integrated, consistent, cleaned data.
    \item Available information-processing structure surrounding data warehouses.
      \begin{itemize}
      \item ODBC, OLEDB, Web access, service facilities, reporting, and OLAP tools.
      \end{itemize}
    \item OLAP-based exploratory data analysis.
      \begin{itemize}
      \item Mining with drilling, dicing, pivoting, etc.
      \end{itemize}
    \item Online selection of data-mining functions.
      \begin{itemize}
      \item Integration and swapping of multiple mining functions, algorithms, and tasks.
      \end{itemize}
    \end{itemize}
  \end{itemize}
\end{frame}
