\section{Data Warehouse: Basic Concepts}

\begin{frame}{What is a Data Warehouse?}
	William ``Bill'' H.  Inmon is commonly refered to as the ``father of the data warehouse''.
	\begin{block}{Data Warehouse}
		A data warehouse is a \textbf{subject-oriented}, \textbf{integrated}, \textbf{time-variant}, and \textbf{nonvolatile} collection of data in support of management's decision-making process.\footnote{\fullcite{inmon2005}} Common abbreviations: DW or DWH.
	\end{block}

	\small
	\textbf{Other definitions exist:}
	\begin{itemize}
		\item ``A data warehouse is a system that extracts, cleans, conforms, and delivers source data into a dimensional data store and then supports and implements querying and analysis for the purpose of decision making.''\footnote{\fullcite{kimball2004}}
		\item A \textbf{\color{airforceblue}decision-support} database that is
		      \textbf{\color{airforceblue}maintained separately} from the organization's
		      operational database.
		\item Supports information processing by providing a solid platform of
		      \textbf{\color{airforceblue}consolidated, historical data} for analysis.
	\end{itemize}

	\textbf{\color{airforceblue}Data warehousing:} The process of constructing and using data warehouses.
\end{frame}

\begin{frame}{Data Warehouse -- Subject-oriented}
	\begin{itemize}
		\item \textbf{Organized around major subjects.} Such as customer, product,
		      sales.
		\item \textbf{Focusing on the modeling and analysis of data for
				      {\color{airforceblue}decision makers}.} Not on daily operations or
		      transaction processing.
		\item \textbf{Provide a simple and concise view around particular
			      subject issues.} By excluding data that are not useful in the
		      decision-support process.
	\end{itemize}
\end{frame}

\begin{frame}{Data Warehouse -- Integrated}
	\begin{itemize}
		\item \textbf{Constructed by {\color{airforceblue}integrating multiple heterogeneous data sources}.}
		      \begin{itemize}
			      \item Relational databases, flat files, online transaction records, \ldots
		      \end{itemize}
		\item \textbf{Data-cleaning and data-integration techniques are applied.}
		      \begin{itemize}
			      \item Ensure consistency in naming conventions, encoding structures, attribute measures, etc. among different data sources.
			      \item E.g., hotel price: currency, tax, breakfast covered.
			      \item When data is moved to the data warehouse, it is converted.
			      \item ETL -- Extract, Transform, Load.
		      \end{itemize}
	\end{itemize}
\end{frame}

\begin{frame}{Data Warehouse -- Time-Variant}
	\begin{itemize}
		\item \textbf{The {\color{airforceblue}time horizon} for a data warehouse is
				      {\color{airforceblue}significantly longer} than that of operational
			      systems.}
		      \begin{itemize}
			      \item Operational database: current-value data.
			      \item Data warehouse: provide information from a historical perspective, e.g. past $5-10$ years.
		      \end{itemize}
		\item \textbf{Every key structure in the data warehouse contains an
			      element of time, explicitly or implicitly.}
		\item The key of operational data may or may not contain a "time element."
	\end{itemize}
\end{frame}

\begin{frame}{Data Warehouse -- Nonvolatile}
	\begin{itemize}
		\item \textbf{A {\color{airforceblue}physically separate} store of data.}
		      \begin{itemize}
			      \item Transformed from the operational environment.
			      \item By {\color{airforceblue}copying}.
		      \end{itemize}
		\item \textbf{No operational update of data:}
		      \begin{itemize}
			      \item Hence, does not require transaction processing, \\
			            i.e. no logging, recovery, concurrency control, etc.
			      \item Requires only three operations:
			            \begin{enumerate}
				            \item Initial loading of data.
				            \item Refresh (update, often periodically, e.g. over night).
				            \item Access of data.
			            \end{enumerate}
		      \end{itemize}
	\end{itemize}
\end{frame}

\begin{frame}{Data Warehouse Usage}
	\textbf{Three kinds of data warehouse applications.}
	\begin{enumerate}
		\item \textbf{\color{airforceblue}Information processing.}
		      \begin{itemize}
			      \item Supports querying, basic statistical analysis, and \\ reporting using crosstabs, tables, charts and graphs.
		      \end{itemize}
		\item \textbf{\color{airforceblue}Analytical processing.}
		      \begin{itemize}
			      \item Multidimensional analysis of data warehouse data.
			      \item Supports basic OLAP operations such as slicing, dicing, drilling, and pivoting.
		      \end{itemize}
		\item \textbf{\color{airforceblue}Data mining.}
		      \begin{itemize}
			      \item Knowledge discovery from hidden patterns.
			      \item Supports associations, constructing analytical models, performing classification and prediction, and presenting the mining results using visualization tools.
		      \end{itemize}
	\end{enumerate}
\end{frame}


\begin{frame}{Online Transaction Processing vs. Online Analytical Processing}
	\vspace{-0.5em}
	\begin{tabularx}{\textwidth}{|b|X|X|}
		\rowcolor{faugray!62}          & \textbf{OLTP}                                            & \textbf{OLAP}                                                      \\\hline
		\textbf{Users}                 & clerk, IT professional                                   & knowledge worker                                                   \\ \hline
		\textbf{Function}              & day-to-day operations                                    & decision support                                                   \\ \hline
		\textbf{DB Design}             & application-oriented                                     & decision support                                                   \\ \hline
		\textbf{Data}                  & current, up-to-date; detailed, flat relational; isolated & historical; summarized, multidimensional, integrated, consolidated \\ \hline
		\textbf{Usage}                 & repetitive                                               & ad-hoc                                                             \\ \hline
		\textbf{Access}                & read/write; index/hash on primary key                    & lots of scans                                                      \\ \hline
		\textbf{Unit of Work}          & short, simple transaction                                & complex query                                                      \\ \hline
		\textbf{$\#$-Records Accessed} & $~ 10$                                                   & $~ 10^6$                                                           \\ \hline
		\textbf{$\#$-Users}            & $~ 1000$                                                 & $~ 100$                                                            \\ \hline
		\textbf{DB Size}               & $~ 100$ MB to GB                                         & $~ 100$ GB to TB                                                   \\ \hline
		\textbf{Quantification}        & transaction throughput                                   & query throughput, response                                         \\ \hline
	\end{tabularx}

	{\tiny OLTP = Online Transaction Processing, OLAP = Online Analytical Processing}
\end{frame}

\begin{frame}{Why a Separate Data Warehouse?}
	\textbf{High performance for both systems:}
	\begin{itemize}
		\item \textbf{\color{airforceblue}DBMS}: tuned for OLTP; Access methods,
		      indexing concurreny control, recovery.
		\item \textbf{\color{airforceblue}Data Warehouse}: tuned for OLAP; Complex OLAP
		      queries, multidimensional view, consolidation.
	\end{itemize}
	\textbf{Different functions and different data:}
	\begin{itemize}
		\item \textit{Missing data (DBMS):} Decision support (DS) requires \textbf{\color{airforceblue}historical data} which operational DBs do not typically maintain.
		\item \textit{Data consolidation (warehouse):} DS requires \textbf{\color{airforceblue}consolidation} (aggregation, summarization) of data from heterogeneous sources.
		\item \textit{Data quality (warehouse):} Different sources typically use inconsistent data representations, codes and formats which have to be reconciled.
	\end{itemize}

	\begin{alertblock}{Note}
		There are more and more systems which perform OLAP analysis directly on relational databases.
	\end{alertblock}
\end{frame}

\begin{frame}{Three Data Warehouse Models}
	\begin{enumerate}
		\item \textbf{\color{airforceblue}Enterprise Warehouse:}
		      \begin{itemize}
			      \item Collects all of the information about subjects spanning the entire organization.
		      \end{itemize}
		\item \textbf{\color{airforceblue}Data Mart:}
		      \begin{itemize}
			      \item A \textbf{\color{airforceblue}subset} of corporate-wide data that is of value to a \textbf{\color{airforceblue}specific group of users}.
			      \item Typically contains (highly) summarized data.
			      \item Independent vs. dependent (directly from warehouse) data mart.
		      \end{itemize}
		\item \textbf{\color{airforceblue}Virtual Warehouse:}
		      \begin{itemize}
			      \item Also known as \textit{data virtualization}.
			      \item A set of \textbf{\color{airforceblue}views} over operational databases.\\
			            {\scriptsize As an \textit{operational database} are all data sources considered that summarize, serve, and access up-to-date and real-time data. Generally, these are OLTP systems that provide ACID properties. These systems include, but are not limited to relational databases, NoSQL datbases, but also unstructured data.}
			      \item Only some of the possible summary views may be materialized.
		      \end{itemize}
	\end{enumerate}
\end{frame}

\begin{frame}{Extract, Transform, and Load (ETL)}
	\begin{itemize}
		\item \textbf{\color{airforceblue}Extract} Data:
		      \begin{itemize}
			      \item Get data from multiple, heterogeneous, and external sources.
		      \end{itemize}
		\item \textbf{Clean} Data:
		      \begin{itemize}
			      \item Detect errors in the data and rectify them if possible.
		      \end{itemize}
		\item \textbf{\color{airforceblue}Transform} Data:
		      \begin{itemize}
			      \item Convert data from legacy or host format to warehouse format.
		      \end{itemize}
		\item \textbf{\color{airforceblue}Load} Data:
		      \begin{itemize}
			      \item Sort, summarize, consolidate, compute views, check integrity, and build indexes and partitions.
		      \end{itemize}
		\item \textbf{Refresh} Data:
		      \begin{itemize}
			      \item Propagate only the updates from the data sources to the warehouse.
		      \end{itemize}
	\end{itemize}
\end{frame}

\begin{frame}{Metadata\footnote{C.f. chapter 9 of \fullcite{kimball2004}} Repository}
	\begin{columns}
		\begin{column}{0.5\textwidth}
			Generally speaking:

			\begin{block}{Metadata}
				Data about data.
			\end{block}

			Three types: \textit{business}, \textit{process execution}, and
			\textit{technical} metadata.

			\vspace*{0.8em}
			\textbf{Business Metadata}
			\begin{itemize}
				\item Business terms and definitions.
				\item Logical data mapping.
				\item Data ownership.
				\item Charging policies.
			\end{itemize}
		\end{column}

		\begin{column}{0.5\textwidth}
			\textbf{Process Execution Metadata}
			\begin{itemize}
				\item Data acquisition schedule.
				\item Data-cleaning specifications.
				\item Aggregate specifications.
				\item Slowly changing dimensions policies.
				\item Duration of ETL, rows rejected and successfull.
			\end{itemize}
			\textbf{Technical Metadata}
			\begin{itemize}
				\item Table structures and table attributes.
				\item Derived data definitions.
				\item Results from data profiling.
				\item Data lineage.
			\end{itemize}

			\note{Data lineage = mapping of data and their tranformation process back
				to their originated source.}
		\end{column}
	\end{columns}
\end{frame}

% TODO: Insert (reference) architecture for data warehouse systems
\begin{frame}{Data Warehouse Reference Overview}
	\begin{tikzpicture}[
		scale=0.8,
		every node/.style={transform shape},
		>=latex,
		thick,
		node distance=2em,
		database/.style={
				cylinder,
				cylinder uses custom fill,
				shape border rotate=90,
				aspect=0.25,
				draw,
				align=center
			}
	]
	\node[database,text width=4em] (source1) at (0,0) {ERP};
	\node[database,below=1em of source1,text width=4em] (source2) {CRM};
	\node[database,below=1em of source2,text width=4em] (source3) {MES};
	\node[database,below=1em of source3,text width=4em] (source4) {\dots};

	\node[signal,minimum width=4em,minimum height=1cm,draw=black,below right=-0.5em and 16em of source2,fill=faured!75] (load) {\hspace*{1.5em}Load};
	\node[signal,minimum width=4em,minimum height=1cm,draw=black,left=-1.6em of load,fill=fauorange] (transform) {\hspace*{1.5em}Transform};
	\node[signal,minimum width=4em,minimum height=1cm,draw=black,left=-1.6em of transform,fill=fauyellow] (extract) {Extract};

	\node[database,right=3em of load,text width=10em,fill=faucyan!75] (dwh) {\huge Data\\\smallskip\huge Warehouse};

	\node[database,right=5em of dwh,text width=4em,fill=faucyan!12] (dm2) {Data Mart 2};
	\node[database,above=2em of dm2,text width=4em,fill=faucyan!12] (dm1) {Data Mart 1};
	\node[database,below=2em of dm2,text width=4em,fill=faucyan!12] (dm3) {Data Mart $n$};

	\draw[->] ([yshift=1mm]source1.east) to[out=0,in=180] (extract.west);
	\draw[->] ([yshift=1mm]source2.east) to[out=0,in=180] (extract.west);
	\draw[->] ([yshift=1mm]source3.east) to[out=0,in=180] (extract.west);
	\draw[->] ([yshift=1mm]source4.east) to[out=0,in=180] (extract.west);

	\draw[->] (load) to (dwh);

	\draw[->] (dwh.east) to[out=0,in=180] (dm1.west);
	\draw[->] (dwh.east) to[out=0,in=180] (dm2.west);
	\draw[->] (dwh.east) to[out=0,in=180] (dm3.west);

	\node[above right=-2.8em and 6em of dm1] (analytics1) {\huge\faChartPie};
	\node[below=2em of analytics1] (analytics2) {\huge\faChartLine};
	\node[below=2em of analytics2] (analytics3) {\huge\faChartBar};
	\node[below=2em of analytics3] (analytics4) {\huge\faChartArea};

	\foreach \i in {1,...,4}
		{
			\foreach \j in {2,...,3}
				{
					\draw[gray,->] (dm\j) to[out=0,in=180] (analytics\i);
				}
		}

	\foreach \i in {1,...,4}
		{
			\draw[->] (dm1) to[out=0,in=180] (analytics\i);
		}
\end{tikzpicture}

\end{frame}
