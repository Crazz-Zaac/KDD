\section{Bayes Classification Methods}

\begin{frame}{Bayesian Classification: Why?}
	\begin{itemize}
		\item \textbf{A statistical classifier:}
		      \begin{itemize}
			      \item Performs probabilistic prediction, i.e. predicts class-membership probabilities.
		      \end{itemize}
		\item \textbf{Foundation:} \textbf{\color{airforceblue}Bayes' Theorem.}
		\item \textbf{Performance:}
		      \begin{itemize}
			      \item A simple Bayesian classifier (naïve Bayesian classifier) has performance comparable with decision tree and selected neural-network classifiers.
		      \end{itemize}
		\item \textbf{Incremental:}
		      \begin{itemize}
			      \item Each training example can incrementally increase/decrease the probability that a hypothesis is correct -- prior knowledge can be combined with observed data.
		      \end{itemize}
		\item \textbf{Standard:}
		      \begin{itemize}
			      \item Even when Bayesian methods are computationally intractable, they can provide a standard of optimal decision making against which other methods can be measured.
		      \end{itemize}
	\end{itemize}
\end{frame}

\begin{frame}{Bayes' Theorem: Basics}
	\begin{itemize}
		\item \textbf{Let $X$ be a data sample ("evidence").}
		      \begin{itemize}
			      \item The class label shall be unknown.
		      \end{itemize}
		\item \textbf{Let $C_i$ be the hypothesis that $X$ belongs to class $i$.}
		\item \textbf{Classification is to determine $P(C_i|X)$:}
		      \begin{itemize}
			      \item \textbf{\color{airforceblue}Posteriori probability:} the probability that the hypothesis \\ holds given the observed data sample $X$.
		      \end{itemize}
		\item $P(C_i)$:
		      \begin{itemize}
			      \item \textbf{\color{airforceblue}Prior probability:} the initial probability.
			      \item E.g., $X$ will buy computer, regardless of age, income, $\ldots$
		      \end{itemize}
		\item $P(X)$:
		      \begin{itemize}
			      \item Probability that sample data is observed.
		      \end{itemize}
		\item $P(X|C_j)$:
		      \begin{itemize}
			      \item \textbf{\color{airforceblue}Likelihood:} the probability of observing the sample $X$ given that the hypothesis holds.
			      \item E.g., given that $X$ buys computer, the probability that $X$ is $31\ldots40$, medium income.
		      \end{itemize}
	\end{itemize}
\end{frame}

\begin{frame}{Bayes' Theorem (II)}
	\begin{itemize}
		\item \textbf{Given training data $X$, the posteriori probability $P(C_i|X)$\\
			      of a hypothesis $C_i$ follows from the Bayes' Theorem:}
		      \begin{align}
			      P(C_i|X) = \frac{P(X|C_i)P(C_i)}{P(X)}.
		      \end{align}
		\item \textbf{Predicts that $X$ belongs to $C_i$ if the probability $P(C_i|X)$\\
			      is {\color{airforceblue}the highest} among all the $P(C_k|X)$ for all $k$ classes.}
		\item \textbf{Practical difficulty:}
		      \begin{itemize}
			      \item Requires initial knowledge of many probabilities.
			      \item Significant computational cost.
		      \end{itemize}
	\end{itemize}
\end{frame}

\begin{frame}{Towards Naïve Bayesian Classifier}
	\begin{itemize}
		\item \textbf{Let $D$ be a training set of tuples and their associated class labels.}
		      \begin{itemize}
			      \item Each tuple is represented by an $n$-dimensional attribute $X = (x_1,x_2,\ldots,x_n)$.
		      \end{itemize}
		\item \textbf{Supose there are $m$ classes $C_1,C_2, \ldots, C_m$.}
		\item \textbf{Classification is to derive the {\color{airforceblue}maximum posteriori probability}.}
		      \begin{itemize}
			      \item i.e. the maximal $P(C_i|X)$.
		      \end{itemize}
		\item \textbf{This can be derived from Bayes' Theorem:}
		      \begin{align}
			      P(C_i|X) = \frac{P(X|C_i)P(C_i)}{P(X)}.
		      \end{align}
		\item \textbf{Since $P(X)$ is constant for all classes, we must maximize only:}
		      \begin{align}
			      P(X|C_i)P(C_i).
		      \end{align}
	\end{itemize}
\end{frame}

\begin{frame}{Derivation of Naïve Bayes Classifier}
	\begin{itemize}
		\item \textbf{A simplifying assumption: attributes are {\color{airforceblue}conditionally independent}.}
		      \begin{itemize}
			      \item I.e. no dependence relation between attributes (which is "naïve").
			            \begin{align*}
				            \resizebox{7cm}{!}{%
					            $P(X|C_i) = \prod_{k=1}^{n} P(x_k|C_i) = P(x_1|C_i)P(x_2|C_i)\cdots P(x_n|C_i).$}
			            \end{align*}
			      \item This greatly reduces the computation cost:\\
			            Only count the class distribution.
			      \item If $A_k$ is categorical,
			            \begin{itemize}
				            \item $P(x_k|C_i)$is the number of tuples in $C_i$ having value $x_k$ for $A_k$ \\
				                  divided by $|C_{i,D}|$ (the number of tuples of $C_i$ in $D$).
			            \end{itemize}
			      \item If $A_k$ is continuous-valued,
			            \begin{itemize}
				            \item $P(x_k|C_i)$ is usually computed based on Gaussian distribution with a mean $\mu$ and standard deviation $\sigma$:
				                  \begin{align*}
					                  \resizebox{4cm}{!}{%
					                  $G(x,\mu,\sigma) = \frac{1}{\sqrt{2\pi}\sigma}e^{\frac{(x-\mu)^2}{2\sigma^2}},$}
				                  \end{align*}
				            \item and $P(x_k|C_i) = G(x_k,\mu_{C_i},\sigma_{C_i})$.
			            \end{itemize}
		      \end{itemize}
	\end{itemize}
\end{frame}

\begin{frame}{Naïve Bayesian Cellcolormake Dataset}
	\begin{columns}
		\begin{column}{0.4\textwidth}
			\vspace{-2cm}
			\begin{itemize}
				\item \textbf{Classes:}
				      \begin{itemize}
					      \item $C_1$: \texttt{buys\_computer} = "yes".
					      \item $C_2$: \texttt{buys\_computer} = "no".
				      \end{itemize}
				\item \textbf{Data sample:}
				      \begin{itemize}
					      \item $X = (\texttt{age} \leq 30,$ \\
					            $\texttt{income} = "medium",$ \\
					            $\texttt{student} = "yes",$\\
					            $\texttt{credit\_rating} = "fair")$.
				      \end{itemize}
			\end{itemize}
		\end{column}
		\begin{column}{0.6\textwidth}
			\resizebox{\columnwidth}{!}{%
				\begin{tabular}{|l|l|c|c|c|}
	\hline
	\rowcolor{faugray!62}\textbf{age} & \textbf{income} & \textbf{student} & \textbf{credit\_rating} & \textbf{buys\_computer} \\\hline
	$\leq 30$                         & high            & no               & fair                    & {\color{faured}no}      \\\hline
	$\leq 30$                         & high            & no               & excellent               & {\color{faured}no}      \\\hline
	$31\ldots40$                      & high            & no               & fair                    & {\color{faugreen}yes}   \\\hline
	$>40$                             & medium          & no               & fair                    & {\color{faugreen}yes}   \\\hline
	$>40$                             & low             & yes              & fair                    & {\color{faugreen}yes}   \\\hline
	$>40$                             & low             & yes              & excellent               & {\color{faured}no}      \\\hline
	$31\ldots40$                      & low             & yes              & excellent               & {\color{faugreen}yes}   \\\hline
	$\leq 30$                         & medium          & no               & fair                    & {\color{faured}no}      \\\hline
	$\leq 30$                         & low             & yes              & fair                    & {\color{faugreen}yes}   \\\hline
	$>40$                             & medium          & yes              & fair                    & {\color{faugreen}yes}   \\\hline
	$\leq 30$                         & medium          & yes              & excellent               & {\color{faugreen}yes}   \\\hline
	$31\ldots40$                      & medium          & no               & excellent               & {\color{faugreen}yes}   \\\hline
	$31\ldots40$                      & high            & yes              & fair                    & {\color{faugreen}yes}   \\\hline
	$>40$                             & medium          & no               & excellent               & {\color{faured}no}      \\\hline
\end{tabular}

			}
		\end{column}
	\end{columns}
\end{frame}

\begin{frame}{Naïve Bayesian Classifier: An Example}
	\begin{itemize}
		\item $P(C_i)$:
		      \begin{itemize}
			      \item $P(\texttt{buys\_computer} = "yes") = \frac{9}{14} = 0.643$.
			      \item $P(\texttt{buys\_computer} = "no") = \frac{5}{14} = 0.357$.
		      \end{itemize}
		\item $X = (\texttt{age} \leq 30 , \texttt{income} = "medium", \texttt{student} = "yes", \texttt{credit\_rating} = "fair")$.
		\item \textbf{Compute $P(X|C_i)$ for each class:}
		      \begin{itemize}
			      \item $P(\texttt{age} \leq 30 | \texttt{buys\_computer} = "yes") = \frac{2}{9} = 0.222$.
			      \item $P(\texttt{age} \leq 30 | \texttt{buys\_computer} = "no") = \frac{3}{5} = 0.6$.
			      \item $P(\texttt{income} = "medium" | \texttt{buys\_computer} = "yes") = \frac{4}{9} = 0.444$.
			      \item $P(\texttt{income} = "medium" | \texttt{buys\_computer} = "no") = \frac{2}{5} = 0.4$.
			      \item $P(\texttt{student} = "yes" | \texttt{buys\_computer} = "yes") = \frac{5}{9} = 0.556$.
			      \item $P(\texttt{student} = "yes" | \texttt{buys\_computer} = "no") = \frac{1}{5} = 0.2$.
			      \item $P(\texttt{credit\_rating} = "fair" | \texttt{buys\_computer} = "yes") = \frac{6}{9} = 0.667$.
			      \item $P(\texttt{credit\_rating} = "fair" | \texttt{buys\_computer} = "no") = \frac{2}{5} = 0.4$.
		      \end{itemize}
	\end{itemize}
\end{frame}

\begin{frame}{Naïve Bayesian Classifier: An Example (II)}
	\begin{itemize}
		\item $P(C_i)$:
		      \begin{itemize}
			      \item $P(X | \texttt{buys\_computer} = "yes") = 0.222 \cdot 0.444 \cdot 0.556 \cdot 0.667 = 0.037$.
			      \item $P(X | \texttt{buys\_computer} = "no") = 0.6 \cdot 0.4 \cdot 0.2 \cdot 0.4 = 0.019$.
		      \end{itemize}
		\item $P(X | C_i) \cdot P(C_i)$:
		      \begin{itemize}
			      \item $P(X | \texttt{buys\_computer} = "yes") \cdot  P(\texttt{buys\_computer} = "yes") = 0.024$.
			      \item $P(X | \texttt{buys\_computer} = "no") \cdot  P(\texttt{buys\_computer} = "no") = 0.007$.
		      \end{itemize}
		\item \textbf{Therefore, $X$ belongs to class $C_1$ (\texttt{buys\_computer} = "yes")}.
	\end{itemize}
\end{frame}

\begin{frame}{Avoiding the Zero-probability Problem}
	\begin{itemize}
		\item \textbf{Naïve Bayesian prediction requires each conditional probability to be non-zero.}
		      \begin{itemize}
			      \item Otherwise, the predicted probability will be zero.
			            \begin{align}
				            \resizebox{4cm}{!}{
					            $P(X|C_i) = \prod_{k=1}^{n} P(x_k|C_i).$}
			            \end{align}
		      \end{itemize}
		\item \textbf{Example:}
		      \begin{itemize}
			      \item Suppose a dataset with $1000$ tuples, \texttt{income} = "low" $(0)$, \texttt{income} = "medium" $(990)$, and \texttt{income} = "high" $(10)$.
		      \end{itemize}
		\item \textbf{Use {\color{airforceblue}Laplacian correction} (or Laplacian estimator):}
		      \begin{itemize}
			      \item Add $1$ to each case:
			            \begin{itemize}
				            \item $P(\texttt{income} = "low") = \frac{1}{1003}$.
				            \item $P(\texttt{income} = "medium") = \frac{991}{1003}$.
				            \item $P(\texttt{income} = "high") = \frac{11}{1003}$.
			            \end{itemize}
			      \item The "corrected" probability estimates are close to their "uncorrected" counterparts.
		      \end{itemize}
	\end{itemize}
\end{frame}

\begin{frame}{Naïve Bayesian Classifier: Comments}
	\textbf{Advantages}
	\begin{itemize}
		\item Easy to implement.
		\item Good results obtained in most of the cases.
	\end{itemize}

	\textbf{Disadvantages}
	\begin{itemize}
		\item Assumption: class conditional independence, therefore loss of accuracy.
		\item Practically, \textbf{dependencies} exist among variables.
		      \begin{itemize}
			      \item E.g., hospital patients:
			            \begin{itemize}
				            \item Profile: age, family history, etc.
				            \item Symptoms: fever, cough, etc.
				            \item Disease: lung cancer, diabetes, etc.
			            \end{itemize}
			      \item Cannot be modeled by naïve Bayesian classifier.
		      \end{itemize}
	\end{itemize}

	\textbf{How to deal with these dependencies?} $\rightarrow$ Bayesian belief networks (see textbook).
\end{frame}
