\section{Data Objects and Attribute Types}

\begin{frame}{Types of Data Sets}
  \begin{columns}
    \begin{column}{0.45\textwidth}
      \textbf{Records:}
      \begin{itemize}[noitemsep]
      \item Relational records.
      \item Data matrix, e.g. numerical matrix, crosstabs.
      \item Document data: text documents, \textbf{term-frequency vectors}. \tikzmark{n1}
      \item \textbf{Transaction data}. \tikzmark{n2}
      \end{itemize}
      \textbf{Graph and network:}
      \begin{itemize}[noitemsep]
      \item World wide web.
      \item Social of information networks.
      \item Molecular structures.
      \end{itemize}
    \end{column}

    \begin{column}{0.45\textwidth}  %%<--- here
      \begin{table}
        \begin{tabular}{|c|c|c|c|c|c|c|}
          \multicolumn{1}{c|}{} & \rotatebox[origin=c]{270}{team} & \rotatebox[origin=c]{270}{couch} & \rotatebox[origin=c]{270}{play} & \rotatebox[origin=c]{270}{ball} & \rotatebox[origin=c]{270}{score} & \rotatebox[origin=c]{270}{game} \\ \hline
          \tikzmark{t1} Document1 & 3 & 0 & 5 & 0 & 2 & 6 \\ \hline
          Document2 & 0 & 7 & 0 & 2 & 1 & 0 \\ \hline
          Document3 & 0 & 1 & 0 & 0 & 1 & 2 \\
          \hline
        \end{tabular}\\[0.5cm]
        \begin{tabular} { | c | l |}
          \hline
          \textbf{TID} & \textbf{Items} \\
          \hline
          \tikzmark{t2} 1 & Bread, Coke, Milk\\
          2 & Beer, Bread\\
          3 & Beer, Coke, Diapers, Milk\\
          4 & Beer, Bread, Diapers, Milk\\
          5 & Coke, Diapers, Milk \\
          \hline
        \end{tabular}
      \end{table}
      \begin{tikzpicture}[remember picture,overlay]
        \path[draw=blue,thick,->]<1-> ([yshift=1mm]n1) -- (t1);
        \path[draw=blue,thick,->]<1-> ([yshift=1mm]n2) -- (t2);
      \end{tikzpicture}
    \end{column}
  \end{columns}
\end{frame}

\begin{frame}{Types of Data Sets}
  \textbf{Ordered data:}
  \begin{itemize}[noitemsep]
  \item Video data: sequences of images.
  \item Temporal data: time series.
  \item Sequential data: transaction sequences.
  \item Genetic sequence data.
  \end{itemize}
  \textbf{Spatial, image and multimedia:}
  \begin{itemize}[noitemsep]
  \item Spatial data: maps.
  \item Image data.
  \item Video data.
  \end{itemize}
\end{frame}

\begin{frame}{Important Characteristics of Structured Data}
  \textbf{Dimensionality}:\\
  Curse of dimensionality (sparse high-dimensional data spaces).\\[0.2cm]
  % make the example with the volume of a cube

  \textbf{Sparsity}:\\
  Only presence counts.\\[0.2cm]

  \textbf{Resolution}:\\
  Patterns depend on the scale.\\[0.2cm]

  \textbf{Distribution}:\\
  Centrality and dispersion.
\end{frame}

\begin{frame}{Data Objects}
  \textbf{Data sets are made up of data objects}.\\
  \textbf{A data object represents an entity}.\\[0.2cm]

  Examples:
  \begin{itemize}
  \item Sales database: customers, store items, sales.
  \item Medical database: patients, treatments.
  \item University database: students, professors, courses.
  \end{itemize}

  They are also called:\\
  Sampels, examples, instances, data points, objects, tuples, \ldots\\[0.2cm]

  \textbf{Data objects are described by attributes}:
  \begin{itemize}
  \item Database rows $\rightarrow$ data objects.
  \item Columns $\rightarrow$ attributes.
  \end{itemize}
\end{frame}

\begin{frame}{Attributes}
  \textbf{Attribute}:\\
  Sometimes also in other context: field, dimension, feature, variable, \ldots\\[0.2cm]
  A data field encodes the property of an entity or feature of a data object.\\
  E.g. \texttt{customer\_ID}, \texttt{name}, \texttt{address}.\\[0.5cm]

  \textbf{Types}:
  \begin{itemize}
  \item Nominal.
  \item Binary.
  \item Ordinal.
  \item Numerical:
    \begin{itemize}
    \item Interval scaled.
    \item Ratio scaled.
    \end{itemize}
  \end{itemize}
\end{frame}

\begin{frame}{Attribute Types}
  \begin{itemize}
  \item \textbf{Nominal}:
    \begin{itemize}
    \item Categories, states, or "names of things".\\
      E.g. \texttt{hair\_color} $= \{\text{auburn, black, blond, brown, grey, red, white}\}$.\\
      Other examples: \texttt{marital\_status}, \texttt{occupation}, \texttt{ID}, \texttt{ZIP code}.
    \end{itemize}
  \item \textbf{Binary}:
    \begin{itemize}
    \item Nominal attribute with only two states ($0$ and $1$).
    \item \textbf{Symmetric binaries}: both outcomes equally important, such as \texttt{gender}.
    \item \textbf{Asymmetric binary}: outcomes not equally important. \\
      E.g. medical test (positive vs. negative).\\
      Convention: assign 1 to most important outcome (e.g. HIV positive).
    \end{itemize}
  \item \textbf{Ordinal}:
    \begin{itemize}
    \item Values have a meaningful order (ranking),\\
      but magnitude between successive values is not known.\\
      E.g. \texttt{size} $= \{\text{small, medium, large}\}$, grades, army rankings.
    \end{itemize}
  \end{itemize}
\end{frame}

\begin{frame}{Numerical Attribute Types}
  \begin{itemize}
  \item \textbf{Numerical: Quantity (integer- or real-valued)}.
  \item \textbf{Interval scaled}:
    \begin{itemize}
    \item Measured on a scale of \textbf{equally sized} units.
    \item Values have order.\\
      E.g. temperature in $C$ or $F$, calender dates.
    \item No true zero-point.
    \end{itemize}
  \item \textbf{Ratio scaled}:
    \begin{itemize}
    \item Inherent \textbf{zero point}.
    \item We can speak of values as being an order of magnitude larger \\
      than the unit of measurement.\\
      E.g. $10 K$ is twice as high as $5 K$.\\
      E.g. temperature in Kelvin, length, counts, monetary quantities.
    \end{itemize}
  \end{itemize}
\end{frame}

\begin{frame}{Discrete vs. Continuous Attributes }
  \begin{itemize}
  \item \textbf{Discrete attribute}:
    \begin{itemize}
    \item Has finite or countably infinite elements.\\
      E.g. ZIP code, profession, or the set of words in a collection of documents.
    \item Sometimes represented as integer variables.
    \item Note: Binary attributes are a special case of discrete attributes.
    \end{itemize}
  \item \textbf{Continuous attribute}:
    \begin{itemize}
    \item Has real numbers as attribute values.\\
      E.g. temperature, height, or weight.
    \item Practically, real values can only be measured and represented using a finite number of digits.
    \item Continuous attributes are typically represented as floating-point variables.
    \end{itemize}
  \end{itemize}
\end{frame}
